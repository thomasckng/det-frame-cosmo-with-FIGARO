\documentclass[aps,prd,twocolumn,superscriptaddress,preprintnumbers,nofootinbib,hidelinks]{revtex4-2}

\usepackage{showyourwork}
\usepackage{amsfonts,amssymb,amsmath}
\usepackage[nolist,nohyperlinks]{acronym}
\usepackage{bookmark}
\usepackage{orcidlink}
\usepackage{mathtools}

\newcommand{\todo}[1]{\textcolor{red}{[TODO: #1]}}

\begin{document}

\title{Detector-frame cosmology with non-parametric methods}

\author{Thomas~C.~K.~Ng\,\orcidlink{0000-0002-9491-1598}}
\email{thomas.ng@link.cuhk.edu.hk}
\affiliation{Department of Physics, The Chinese University of Hong Kong, Shatin, Hong Kong}

\author{Stefano~Rinaldi\,\orcidlink{0000-0001-5799-4155}}
\email{stefano.rinaldi@uni-heidelberg.de}
\affiliation{Institut~für~Theoretische~Astrophysik, ZAH, Universität~Heidelberg, Albert-Ueberle-Stra{\ss}e~2, 69120 Heidelberg, Germany}
\affiliation{Dipartimento di Fisica e Astronomia ``G. Galilei'', Università di Padova, Via Marzolo 8, 35122 Padova, Italy}

\author{Otto~A.~Hannuksela\,\orcidlink{0000-0002-3887-7137}}
\email{hannuksela@phy.cuhk.edu.hk}
\affiliation{Department of Physics, The Chinese University of Hong Kong, Shatin, Hong Kong}

\begin{abstract}
    The challenge of understanding the Universe's dynamics, particularly the Hubble tension, requires precise measurements of the Hubble constant.
    Building upon the existing spectral siren method, which capitalizes on population information from gravitational wave sources, this study explores an alternative way to analyze the population data to obtain the cosmological parameters in $\Lambda$CDM.
    We will focus on how non-parametric methods, which are flexible models that can be used to agnostically reconstruct arbitrary probability densities, can be incorporated into this framework and leverage the detector-frame mass distribution to infer the cosmological parameters.
    \todo{Rewrite abstract}
\end{abstract}

\maketitle

\begin{acronym}
    \acro{GW}{gravitational wave}
    \acro{LVK}{LIGO-Virgo-KAGRA Collaboration}
    \acro{PE}{parameter estimation}
    \acro{CMB}{cosmic microwave background}
    \acro{EM}{electromagnetic}
\end{acronym}

\section{Introduction}
\label{sec:introduction}

% Motivation
In recent years, the precision of cosmological measurements has improved significantly, leading to the discovery of the Hubble tension, a discrepancy between the value of the Hubble constant $H_0$ inferred from the \ac{CMB} \citep{Planck:2018vyg} and local measurements \citep{Riess:2021jrx}.
This tension has motivated the search for new methods to measure $H_0$ with high precision, and \ac{GW} sources have emerged as a promising tool for this purpose.
To measure $H_0$ from \ac{GW} sources, one needs not only the luminosity distance which can be inferred from the signal, but also the redshift of the source.
Some \ac{GW} sources can be associated with \ac{EM} counterparts, allowing for the measurement of the redshift. \citep{LIGOScientific:2017adf}
    % Without EM counterpart, can measure redshift from population statistics

% Previous work
    % Spectral siren
    % Other population studies

% What's new?
    % Non-parametric methods
    % Detector-frame

% Section outline
In Sec.~\ref{sec:method},
In Sec.~\ref{sec:results},
In Sec.~\ref{sec:discussion},
In Sec.~\ref{sec:conclusion},

\section{Method}
\label{sec:method}

In this section, we describe the framework of our method, 

\subsection{Framework}
\label{sec:framework}

\subsubsection{Non-parametric reconstruction of detector-frame observed population}
\label{sec:reconstruction}

\subsubsection{Transformation of source-frame population model}
\label{sec:transformation}

\subsubsection{Inference}
\label{sec:inference}

\subsection{Injection test}
\label{sec:injection}

\section{Results}
\label{sec:results}

\section{Discussion}
\label{sec:discussion}

\section{Conclusion}
\label{sec:conclusion}

\begin{acknowledgments}

S.R.~acknowledges financial support from the European Research Council for the ERC Consolidator grant DEMOBLACK, under contract no.~770017.
This paper was compiled using \textsc{showyourwork} \cite{Luger2021} to facilitate reproducibility.

\end{acknowledgments}

\bibliography{bib}

\end{document}
