\documentclass[aps,prd,twocolumn,superscriptaddress,preprintnumbers,nofootinbib,hidelinks]{revtex4-2}

\usepackage{showyourwork}
\usepackage{amsfonts,amssymb,amsmath}
\usepackage[nolist,nohyperlinks]{acronym}
\usepackage{bookmark}
\usepackage{orcidlink}
\usepackage{mathtools}

\newcommand{\todo}[1]{\textcolor{red}{[TODO: #1]}}

\begin{document}

\title{Detector-frame cosmology with non-parametric methods}

\author{Thomas~C.~K.~Ng\,\orcidlink{0000-0002-9491-1598}}
\email{thomas.ng@link.cuhk.edu.hk}
\affiliation{Department of Physics, The Chinese University of Hong Kong, Shatin, Hong Kong}

\author{Stefano~Rinaldi\,\orcidlink{0000-0001-5799-4155}}
\email{stefano.rinaldi@uni-heidelberg.de}
\affiliation{Institut~für~Theoretische~Astrophysik, ZAH, Universität~Heidelberg, Albert-Ueberle-Stra{\ss}e~2, 69120 Heidelberg, Germany}
\affiliation{Dipartimento di Fisica e Astronomia ``G. Galilei'', Università di Padova, Via Marzolo 8, 35122 Padova, Italy}

\author{Otto~A.~Hannuksela\,\orcidlink{0000-0002-3887-7137}}
\email{hannuksela@phy.cuhk.edu.hk}
\affiliation{Department of Physics, The Chinese University of Hong Kong, Shatin, Hong Kong}

\begin{abstract}
    The challenge of understanding the Universe's dynamics, particularly the Hubble tension, requires precise measurements of the Hubble constant.
    Building upon the existing spectral siren method, which capitalizes on population information from gravitational wave sources, this study explores an alternative way to analyze the population data to obtain the cosmological parameters in $\Lambda$CDM.
    We will focus on how non-parametric methods, which are flexible models that can be used to agnostically reconstruct arbitrary probability densities, can be incorporated into this framework and leverage the detector-frame mass distribution to infer the cosmological parameters.
    \todo{Rewrite abstract}
\end{abstract}

\maketitle

\begin{acronym}
    \acro{GW}{gravitational wave}
    \acro{LVK}{LIGO-Virgo-KAGRA Collaboration}
    \acro{PE}{parameter estimation}
    \acro{CMB}{cosmic microwave background}
    \acro{EM}{electromagnetic}
    \acro{DPGMM}{Dirichlet process Gaussian mixture model}
    \acro{BBH}{binary black hole}
    \acro{GWTC-3}{the third Gravitational-Wave Transient Catalog}
\end{acronym}

\section{Introduction}
\label{sec:introduction}

% Motivation
In recent years, the precision of cosmological measurements has improved significantly, leading to the discovery of the Hubble tension, a discrepancy between the value of the Hubble constant $H_0$ inferred from the \ac{CMB} \citep{Planck:2018vyg} and local measurements \citep{Riess:2021jrx}.
This tension has motivated the search for new methods to measure $H_0$ with high precision, and \ac{GW} sources have emerged as a promising tool for this purpose.
To infer $H_0$ from \ac{GW} sources, one needs not only the luminosity distance which can be inferred from the signal, but also the redshift of the source.
Some \ac{GW} sources can be associated with \ac{EM} counterparts, allowing for the measurement of the redshift, e.g., GW170817 \citep{LIGOScientific:2017adf, Guidorzi:2017ogy}.
However, most \ac{GW} sources do not have \ac{EM} counterparts.
In this case, one can use population statistics to infer $H_0$ by marginalizing over the redshift distribution of the sources.
This method is known as the spectral siren \citep{You:2020wju, Mastrogiovanni:2021wsd, LIGOScientific:2021aug, Ezquiaga:2022zkx}.

% Previous work
The fundamental idea of the spectral siren method is to assume a population model for the mass distribution of the sources and marginalize over the redshift distribution to infer the cosmological parameters.
Many studies have explored different variations of the spectral siren method.
\todo{Summarize previous work and add references}

% What's new?
In this study, we explore an alternative method based on the spectral siren framework.
We propose two major modifications to the standard spectral siren method.
First, instead of transforming the luminosity distance to redshift during the \ac{PE} process, we transform the population model to the detector frame.
Second, instead of computing the likelihood of the observed data given the population model, we reconstruct the observed population directly from the data using non-parametric methods.
With these modifications, we aim to separate the spectral siren method into two distinct parts.
The first part is the reconstruction of the observed population from the data, which does not require any assumptions about the population model, and the intermediate result is completely data-driven.
All the assumptions including the population model, cosmological model and the selection function are then incorporated in the second half.
This separation allows us to explore the impact of different assumptions easily.
\todo{Add references}

% Section outline
In Sec.~\ref{sec:method}, we describe our modified method and the testing procedure.
In Sec.~\ref{sec:results}, we present the results of our method.
In Sec.~\ref{sec:discussion}, we compare our result with other methods and discuss future directions.
Finally, we conclude in Sec.~\ref{sec:conclusion} with a summary.
\todo{Rewrite outline}

\section{Method}
\label{sec:method}

In this section, we describe the framework of our method, followed by a description of the testing setup.

\subsection{Framework}
\label{sec:framework}

Our method consists of three main steps: the non-parametric reconstruction of the observed population, the transformation of the source-frame population model to the detector frame, and the \ac{PE}.
We show the flowchart of our method in Fig.~XXX. \todo{Add flowchart}

\subsubsection{Non-parametric reconstruction of detector-frame observed population}
\label{sec:reconstruction}

The first step of our method is to reconstruct the observed population.
we choose to consider the primary mass $m_1$ of the \ac{BBH} sources.
From the posterior samples obtained from the \ac{PE} of each \ac{GW} event, we can reconstruct the observed detector-frame mass distribution $p(m_1|\{d\})$, where $\{d\}$ is the set of all the observed data.

To perform the reconstruction, we use a non-parametric reconstruction package \textsc{FIGARO} \citep{Rinaldi:2022kyg}.
\textsc{FIGARO} can reconstruct an arbitrary probability density function from a set of samples based on the \ac{DPGMM}.
It can also perform a hierarchy of \ac{DPGMM} to reconstruct an polulation distribution with posterior samples from multiple events \citep{Rinaldi:2021bhm}.
The output of \textsc{FIGARO} is a set of distributions, where each distribution is a realization of the observed population that is consistent with the data.
Therefore, the set of distributions represents the uncertainty in the reconstruction procedure.
In Fig.~XXX \todo{Add figure}, we show an example of the reconstruction obtained with \textsc{FIGARO}.

This part of the method is completely data-driven and does not require any assumptions about the population model or the cosmological model.
Therefore the intermediate result is independent of the assumptions, and we can reuse the result for different population models, cosmological models, or selection functions.
This reduces the computational cost of the method, as we only need to perform the reconstruction of the observed population once.

\subsubsection{Transformation of source-frame population model}
\label{sec:transformation}

On the other hand, we need a source-frame population model $p(m_1|\Lambda)$ to represent the underlying population of the sources, where $\Lambda$ is the polulation model.
We can either use a parametric model to only assume the form of the population distribution, or we can use a fixed population model to assume the exact population distribution.
With $p(m_1|\Lambda)$, we need to transform it to the detector frame with a given cosmological model to obtain the detector-frame population model $p(m_1|\Lambda, \Omega)$, where $\Omega$ is the cosmological model.
In this study, we use the $\Lambda$CDM model with the Planck 2018 cosmological parameters \citep{Planck:2018vyg}, except for $H_0$ which we choose to be a free parameter.
With different $\Omega$, $p(m_1|\Lambda)$ will be transformed differently, and the resulting $p(m_1|\Lambda, \Omega)$ will be different.
In Fig.~XXX \todo{Add figure}, we show an example of the transformation without any selection effects.

After the cosmological transformation, we need to consider the selection effects.
The selection effects can be easily incorporated by applying the selection function to the transformed population.
\todo{Add more details. Parametric or fixed?}

Opposite to the first part of the method, which is completely data-driven, this part includes all the assumptions about the population model, the cosmological model, and the selection function.

\subsubsection{Parameter estimation}
\label{sec:pe}

With $p(m_1|\{d\})$ and $p(m_1|\Lambda, \Omega)$ obtained from Sec.~\ref{sec:reconstruction} and Sec.~\ref{sec:transformation} respectively, we can perform the \ac{PE} to infer the free parameters in the parametric models assumed in Sec.~\ref{sec:transformation}.
The \ac{PE} is performed by comparing the observed population $p(m_1|\{d\})$ with the transformed population $p(m_1|\Lambda, \Omega)$.
To compare the two populations, we use the Jensen–Shannon distance $d_{JS}$ as the distance measure.
$d_{JS}$ is calculated as
\begin{equation}
    d_{JS}(p, q) = \sqrt{\frac{D_{KL}(p||m) + D_{KL}(q||m)}{2}},
\end{equation}
where $p$ and $q$ are the two distributions to be compared, $m = (p + q) / 2$, and $D_{KL}$ is the Kullback–Leibler divergence.
$d_{JS}$ is a symmetric and smoothed version of $D_{KL}$, and it is bounded between 0 and 1.
For each reconstructed population $p(m_1|\{d\})$, we find the best-fit parameters by minimizing $d_{JS}$ with respect to the free parameters.
This procedure gives us a set of best-fit parameters for each reconstructed population.
Note that the resulting distribution of the best-fit parameters is not a posterior distribution in the Bayesian sense, as we do not consider the likelihood of the observed data given the population model.
Instead, it is a distribution of the best-fit parameters that are consistent with the observed data, distributed according to the uncertainty in the reconstruction of the observed population.
In Fig.~XXX \todo{Add figure}, we show an example of the fitting procedure.

\subsection{Injection test}
\label{sec:injection}

\subsection{GWTC-3 test}
\label{sec:gwtc3}

\section{Results}
\label{sec:results}

\section{Discussion}
\label{sec:discussion}

% Comparison with other methods

% Future directions
    % Likelihood
    % luminosity distance/redshift polulation 

\section{Conclusion}
\label{sec:conclusion}

\begin{acknowledgments}

S.R.~acknowledges financial support from the European Research Council for the ERC Consolidator grant DEMOBLACK, under contract no.~770017.
This paper was compiled using \textsc{showyourwork} \cite{Luger2021} to facilitate reproducibility.

\end{acknowledgments}

\bibliography{bib}

\end{document}
