\documentclass[aps,prd,twocolumn,superscriptaddress,preprintnumbers,nofootinbib,hidelinks]{revtex4-2}

\usepackage{showyourwork}
\usepackage{amsfonts,amssymb,amsmath}
\usepackage[nolist,nohyperlinks]{acronym}
\usepackage{bookmark}
\usepackage{orcidlink}
\usepackage{mathtools}
\usepackage{array}

\newcommand{\todo}[1]{\textcolor{red}{[TODO: #1]}}
\newcommand{\ste}[1]{\textcolor{blue}{[Ste: #1]}}

\begin{document}

\title{Inferring cosmology from gravitational waves using non-parametric detector-frame mass distribution}

\author{Thomas~C.~K.~Ng\,\orcidlink{0000-0002-9491-1598}}
\email{thomas.ng@link.cuhk.edu.hk}
\affiliation{Department of Physics, The Chinese University of Hong Kong, Shatin, Hong Kong}

\author{Stefano~Rinaldi\,\orcidlink{0000-0001-5799-4155}}
\email{stefano.rinaldi@uni-heidelberg.de}
\affiliation{Institut~für~Theoretische~Astrophysik, ZAH, Universität~Heidelberg, Albert-Ueberle-Stra{\ss}e~2, 69120 Heidelberg, Germany}
\affiliation{Dipartimento di Fisica e Astronomia ``G. Galilei'', Università di Padova, Via Marzolo 8, 35122 Padova, Italy}

\author{Otto~A.~Hannuksela\,\orcidlink{0000-0002-3887-7137}}
\email{hannuksela@phy.cuhk.edu.hk}
\affiliation{Department of Physics, The Chinese University of Hong Kong, Shatin, Hong Kong}

\begin{abstract}
    The challenge of understanding the Universe's dynamics, particularly the Hubble tension, requires precise measurements of the Hubble constant.
    Building upon the existing spectral siren method, which capitalizes on population information from gravitational wave sources, this study explores an alternative way to analyze the population data to obtain the cosmological parameters in $\Lambda$CDM.
    We will focus on how non-parametric methods, which are flexible models that can be used to agnostically reconstruct arbitrary probability densities, can be incorporated into this framework and leverage the detector-frame mass distribution to infer the cosmological parameters.
    \todo{Rewrite abstract}
\end{abstract}

\maketitle

\begin{acronym}
    \acro{GW}{gravitational-wave}
    \acro{LVK}{LIGO-Virgo-KAGRA Collaboration}
    \acro{PE}{parameter estimation}
    \acro{CMB}{cosmic microwave background}
    \acro{EM}{electromagnetic}
    \acro{(H)DPGMM}{hierarchy of Dirichlet process Gaussian mixture models}
    \acro{BBH}{binary black hole}
    \acro{GWTC-3}{the third Gravitational-Wave Transient Catalog}
\end{acronym}

\section{Introduction}
\label{sec:introduction}

% Motivation
In recent years, the precision of cosmological measurements has improved significantly, leading to the discovery of the Hubble tension, a discrepancy between the value of the Hubble constant $H_0$ inferred from the \ac{CMB} \citep{Planck:2018vyg} and local measurements \citep{Riess:2021jrx}.
This tension has motivated the search for new methods to measure $H_0$ with high precision, and \ac{GW} sources have emerged as a promising tool for this purpose.
To infer $H_0$ from \ac{GW} sources, one needs not only the luminosity distance which can be inferred from the signal but also the redshift of the source.
Some \ac{GW} sources can be associated with \ac{EM} counterparts, allowing for the measurement of the redshift, e.g., GW170817 \citep{LIGOScientific:2017adf, Guidorzi:2017ogy}.
However, most \ac{GW} sources do not have \ac{EM} counterparts.
In this case, one can assume \ac{GW} sources are in galaxies and associate the redshift of the galaxy to the \ac{GW} source. \citep{Schutz:1986gp, DelPozzo:2011vcw, Gray:2019ksv, Gray:2023wgj}
Another alternative method is to use population statistics to infer $H_0$ by marginalizing the redshift distribution of the sources.
This method is known as the spectral siren method\citep{You:2020wju, Mastrogiovanni:2021wsd, LIGOScientific:2021aug, Ezquiaga:2022zkx}.

% Previous work
The fundamental idea of the spectral siren method is to infer cosmology by comparing the observed mass population with the intrinsic mass population since the observed mass population is redshifted due to the expansion of the Universe.
Assumptions about the intrinsic population model and cosmological model are required to perform the comparison.
Previous studies often infer either the intrinsic population model or the cosmological model, but not both simultaneously, due to the expensive computational cost.
\todo{Summarize previous work and add references}
However, the intrinsic population model and the cosmological model are degenerate in general, and it is necessary to infer both simultaneously.

% Non-parametric methods
To address this issue, we study an alternative method making use of non-parametric methods.
Non-parametric methods allow us to reconstruct arbitrary probability densities without making assumptions about the form of the distribution.
In our context, we can use non-parametric methods to reconstruct the observed mass population distribution directly from the data \todo{better wording?}.
This provides an intermediate observation-driven result that encodes both the intrinsic population information and the cosmological information.
By separating the reconstruction of the observed population from the inference of the intrinsic population and cosmological model, we can explore different models easily.
Furthermore, the intermediate result represents the information contained in the observed data without any assumptions about the population model, cosmological model, or selection function, which allows us to study the features of the observed population directly.

Some other studies have made use of non-parametric methods with the spectral siren framework.
\todo{Summarize previous work and add references}
\todo{Compare with our work}

\todo{More events in the future}

% Section outline
In Sec.~\ref{sec:method}, we describe our modified method and the testing procedure.
In Sec.~\ref{sec:simulation}, we present the analysis setup and results of the simulation study.
In Sec.~\ref{sec:real_data}, we apply our method to the real data from the \ac{LVK}.
In Sec.~\ref{sec:discussion}, we compare our results with other methods and discuss future directions.
Finally, we conclude in Sec.~\ref{sec:conclusion} with a summary.
\todo{Rewrite outline}

\section{Method}
\label{sec:method}
In this section, we present the framework we developed to infer the cosmological parameters leveraging the detector-frame mass distribution. In the remainder of this Section, we will make use of the notation summarized in Table~\ref{tab:notation}, mutuated from \citet{Rinaldi:2021bhm, Rinaldi:2022kyg}.
\todo{Renew notations}

\begin{table}
    \label{tab:notation}
    \caption{Notation}
    \begin{ruledtabular}
        \begin{tabular}{c>{\raggedright\arraybackslash}p{0.8\linewidth}}
            Notation & Description \\
            \hline
            $m_1$ & Primary mass of the \ac{GW} sources \\
            $z$ & Redshift of the \ac{GW} sources \\
            $d_L$ & Luminosity distance of the \ac{GW} sources \\
            $m^z_1$ & Primary mass of the \ac{BBH} sources in the detector frame \\
            $\mathbf{Y}$ & Set of posterior samples of all \ac{GW} events \\
            $\mathbf{\Theta}$ & Set of hyperparameters of the \ac{(H)DPGMM} \\
            $p(m^z_1|\mathbf{\Theta})$ & Detector-frame observed mass population distribution reconstructed with \ac{(H)DPGMM} \\
            $\Lambda$ & Source-frame population model \\
            $\Omega$ & Cosmological model \\
            $p(m_1|\Lambda)$ & Source-frame mass population distribution assuming the population model $\Lambda$ \\
            $p(m^z_1|\Lambda, \Omega)$ & Detector-frame mass population distribution assuming the population model $\Lambda$ and cosmological model $\Omega$ \\
        \end{tabular}
        \end{ruledtabular}
\end{table}

Specifically, we propose two major modifications to the standard spectral siren method.
First, instead of transforming the individual posterior samples from the detector frame to the source frame during the \ac{PE} process, we transform the population model to the detector frame before performing \ac{PE}.
Second, instead of computing the likelihood of the observed data given the population model, we reconstruct the detector-frame observed population directly from the data using a non-parametric method.

With these modifications, we separate the spectral siren method into two distinct parts.
The first part is the reconstruction of the observed population from the data, which does not require any assumptions about the population model, and the intermediate result is completely data-driven.
We make use of a \ac{(H)DPGMM} developed in \citet{Rinaldi:2021bhm} as a non-parametric model to reconstruct the observed population distribution.
Other assumptions including the population model, cosmological model and selection function are then incorporated in the second half.
This separation allows us to explore the impact of different assumptions easily.
The intermediate result of the first part can be reused as long as the data does not change, reducing the computational cost of the method.

Our method consists of three main steps: the non-parametric reconstruction of the detector-frame observed population, the transformation of the source-frame population model to the detector frame, and the \ac{PE}.
We show the flowchart of our method in Fig.~XXX. \todo{Add flowchart}

\subsection{Non-parametric reconstruction of detector-frame observed population}
\label{sec:reconstruction}

The first step of our method is to reconstruct the detector-frame observed population.
we choose to consider the primary mass $m_1$ of the \ac{BBH} sources, in principle, we can consider more parameters at once.
From the posterior samples obtained from the \ac{PE} of each \ac{GW} event, we reconstruct the detector-frame observed mass distribution.

We use a \ac{(H)DPGMM} developed in \citet{Rinaldi:2021bhm} as a non-parametric model to reconstruct the detector-frame observed mass population distribution $p(m^z_1|\mathbf{\Theta})$.
The \ac{(H)DPGMM} is a hierarchy of Gaussian mixture models with a Dirichlet process prior, which allows the model to be flexible to reconstruct arbitrary population distribution from posterior samples.
To perform the reconstruction, we use the package \textsc{figaro}\footnote{\textsc{figaro} is publicly available at \url{https://github.com/sterinaldi/FIGARO} and via \texttt{pip}.} \citep{Rinaldi:2024eep}.
Performing the reconstruction once will give a distribution $p(m^z_1|\Theta_i)$ that is consistent with $\mathbf{Y}$ with a given set of hyperparameters $\Theta_i$.
To account for the uncertainty in the reconstruction, we need to perform the reconstruction multiple times to obtain a set of distributions $p(m^z_1|\mathbf{\Theta})$, where $\mathbf{\Theta}$ is a set of $\Theta_i$.
The uncertainty of the reconstruction can then be quantified by the spread of the distributions.
It is important to note that the reconstruction is only valid in the region where the samples are present.
In Fig.~XXX \todo{Add figure}, we show an example of the reconstruction obtained with \textsc{figaro}.

This part of the method is completely data-driven.
Therefore the intermediate result is independent of the population, cosmological model and selection function, and we can reuse the result for testing different models.
This reduces the computational cost of the method, as we only need to perform the reconstruction once to analyze different models.

\subsection{Transformation of source-frame population model}
\label{sec:transformation}

On the other hand, we assume a population model $\Lambda$ to obtain a source-frame mass population $p(m_1|\Lambda)$, which represents the underlying population of the sources.
Note that, either a parametric model can be used to only assume the form of the population distribution, or a fixed population model can be used to assume an exact population distribution.
We then transform $p(m_1|\Lambda)$ to the detector frame with a given cosmological model $\Omega$ to obtain the detector-frame mass population $p(m^z_1|\Lambda, \Omega)$ by
\begin{widetext}
    \begin{equation}
        \label{eq:transformation}
        \begin{split}
            &p(m^z_1|\Lambda, \Omega, \mathrm{det}) \\
            &= \int p(m^z_1|m_1, z, \Lambda, \Omega, \mathrm{det}) p(m_1, z|\Lambda, \Omega, \mathrm{det}) \mathrm{d}m_1 \mathrm{d}z \\
            &= \int \frac{p(\mathrm{det}|m^z_1, m_1, z, \Omega)p(m^z_1|m_1, z, \Omega)}{p(\mathrm{det}|m_1, z, \Omega)} \frac{p(\mathrm{det}|m_1, z, \Lambda, \Omega)p(m_1, z|\Lambda, \Omega)}{p(\mathrm{det}|\Lambda, \Omega)} \mathrm{d}m_1 \mathrm{d}z \\
            &= \frac{1}{p(\mathrm{det}|\Lambda, \Omega)}\int \frac{p(\mathrm{det}|m_1, z, \Omega)p(m^z_1|m_1, z)p(\mathrm{det}|m_1, z, \Omega)p(m_1, z|\Lambda)}{p(\mathrm{det}|m_1, z, \Omega)}  \mathrm{d}m_1 \mathrm{d}z \\
            &\propto \int \delta(m^z_1-m_1(1+z))p(\mathrm{det}|m_1, z, \Omega)p(m_1|\Lambda)p(z|\Lambda) \mathrm{d}m_1 \mathrm{d}z \\
        \end{split}
    \end{equation}
\end{widetext}
where we assume the source-frame mass distribution does not evolve with redshift, and $\mathrm{det}$ represents whether the event is detectable or not.
We include the selection effect in Eq.~\eqref{eq:transformation}, which gives us $p(m^z_1|\Lambda, \Omega, \mathrm{det})$ in the final expression.
With different $\Omega$, the resulting $p(m^z_1|\Lambda, \Omega, \mathrm{det})$ will be different.
By considering a continuous set of $\Omega$ and  $\Lambda$ of the population model, we can obtain a set of $p(m^z_1|\mathbf{\Lambda}, \mathbf{\Omega}, \mathrm{det})$, which is a set of continuous detector-frame observed population models for each combination of $\Omega_j$ and $\Lambda_j$, where $\mathbf{\Lambda} = \{\Lambda_j\}$ and $\mathbf{\Omega} = \{\Omega_j\}$.\footnote{In practice, we use a discrete set of $\Omega_j$ and $\Lambda_j$ with a small step size in their hyperparameters to approximate the continuous set.}
\ste{This following sentence should be the very last sentence of this subsection, since it opens for the following PE description.} 
After obtaining $p(m^z_1|\mathbf{\Lambda}, \mathbf{\Omega}, \mathrm{det})$, we can perform the \ac{PE} to infer the optimized $\Omega_j$ and $\Lambda_j$.
\todo{PE\textrightarrow Inference of Cosmological Parameters}
\ste{In the selection function part, you may also want to specify why we opted for reconstructing the observed distribution and not the intrinsic one with (H)DPGMM (namely the censoring of a specific portion of the parameter space).}
In Fig.~XXX \todo{Add figure}, we show an example of the transformation.

Opposite to the first part of the method, this part includes the assumptions about the population and cosmological model, as well as the selection function.

\subsection{Parameter estimation}
\label{sec:pe}

With $p(m^z_1|\mathbf{\Theta}(\mathbf{Y}))$ and $p(m^z_1|\mathbf{\Lambda}, \mathbf{\Omega})$ obtained from Sec.~\ref{sec:reconstruction} and Sec.~\ref{sec:transformation} respectively, we can perform the \ac{PE} to infer the free parameters in the parametric models assumed in Sec.~\ref{sec:transformation}.
The \ac{PE} is performed by comparing $p(m^z_1|\Theta_i(\mathbf{Y}))$ with $p(m^z_1|\Lambda_j, \Omega_j)$.
To compare the two populations, we use the Jensen-Shannon distance $d_{JS}$ as the distance measure.
$d_{JS}$ is calculated as
\begin{equation}
    d_{JS}(p, q) = \sqrt{\frac{D_{KL}(p||m) + D_{KL}(q||m)}{2}},
\end{equation}
where $p$ and $q$ are the two distributions to be compared, $m = (p + q) / 2$, and $D_{KL}$ is the Kullback-Leibler divergence.
$d_{JS}$ is a symmetric and smoothed version of $D_{KL}$, and it is bounded between 0 and 1.

For each reconstructed distribution $p(m^z_1|\Theta_i(\mathbf{Y}))$, we find the optimized $\Lambda_j$ and $\Omega_j$ by minimizing $d_{JS}$.
\todo{add the cutting of the low and high mass}
For the minimization, we use the \textsc{scipy} \citep{2020SciPy-NMeth} implementation of the \todo{XXX} algorithm. \todo{Add reference for the algorithm}
This procedure gives us a set of optimized $\Lambda_i$ and $\Omega_j$ for each $\Theta_i$.
Note that the resulting distribution of the optimized parameters is not a posterior distribution in the Bayesian sense, as we do not consider the likelihood of the observed data given the population model.
Instead, it is a distribution of the optimized parameters that are consistent with the observed data, distributed according to the uncertainty in the reconstruction of the population.
That is, the uncertainty in the comparison of the two populations is not considered in the resulting distribution.
By considering this optimization procedure only, we can estimate the optimized parameters without performing computationally expensive likelihood calculations.
In Fig.~XXX \todo{Add figure}, we show an example of the fitting procedure.

The analysis code is publicly available.\footnote{\todo{Add link}}

\section{Simulation}
\label{sec:simulation}

% $H_0$ only

% $H_0$ and $\Lambda$

\section{Real data}
\label{sec:real_data}

Data from \citet{LIGOScientific:2019lzm, KAGRA:2023pio}

% Power law + peak

% Population model from simulation

\section{Discussion}
\label{sec:discussion}

% Comparison with other methods

% PL+P issue
    % Gregoire Pierra's paper

% Future directions
    % Likelihood
    % luminosity distance/redshift population 

\section{Conclusion}
\label{sec:conclusion}

\begin{acknowledgments}
T.~N.~ and S.~R.~ acknowledge financial support from the German Excellence Strategy via the Heidelberg Cluster of Excellence (EXC 2181 - 390900948) STRUCTURES.
S.~R.~ acknowledges financial support from the European Research Council for the ERC Consolidator grant DEMOBLACK, under contract no. 770017. 

This research made use of the bwForCluster Helix: the authors acknowledge support by the state of Baden-Württemberg through bwHPC and the German Research Foundation (DFG) through grant INST 35/1597-1 FUGG.

This research has made use of data or software obtained from the Gravitational Wave Open Science Center (gwosc.org), a service of the LIGO Scientific Collaboration, the Virgo Collaboration, and KAGRA. This material is based upon work supported by NSF's LIGO Laboratory which is a major facility fully funded by the National Science Foundation, as well as the Science and Technology Facilities Council (STFC) of the United Kingdom, the Max-Planck-Society (MPS), and the State of Niedersachsen/Germany for support of the construction of Advanced LIGO and construction and operation of the GEO600 detector. Additional support for Advanced LIGO was provided by the Australian Research Council. Virgo is funded, through the European Gravitational Observatory (EGO), by the French Centre National de Recherche Scientifique (CNRS), the Italian Istituto Nazionale di Fisica Nucleare (INFN) and the Dutch Nikhef, with contributions by institutions from Belgium, Germany, Greece, Hungary, Ireland, Japan, Monaco, Poland, Portugal, Spain. KAGRA is supported by Ministry of Education, Culture, Sports, Science and Technology (MEXT), Japan Society for the Promotion of Science (JSPS) in Japan; National Research Foundation (NRF) and Ministry of Science and ICT (MSIT) in Korea; Academia Sinica (AS) and National Science and Technology Council (NSTC) in Taiwan.

This paper was compiled using \textsc{showyourwork} \cite{Luger2021} to facilitate reproducibility.

\end{acknowledgments}

\bibliography{bib}

\end{document}
